\documentclass[../main.tex]{subfiles}


\begin{document}


\appendix
\chapter{Beviser}
\section{Sum af linearkombinationer af vektorer med længde 1}
\begin{theorem}\label{app:vsum}
	Lad $ \mathbf u, \mathbf v\in\RR^n $ være to vektorer, der begge summer til 1.
	Da gælder for alle $ a+b=1, a, b\in\RR $, at
	\begin{equation*}
		\sum_{i=1}^n(a\mathbf u+b\mathbf v)=1
	\end{equation*}
\end{theorem}
\begin{proof}
	Konvergente summer er lineære.
	Det gælder derfor, at
	\begin{equation*}
		\sum_{i=1}^n(a\mathbf u+b\mathbf v)
		=a\sum_{i=1}^n \mathbf u+b\sum_{i=1}^n \mathbf v
	\end{equation*}
	Pr. antagelserne er begge summer 1.
	Det efterlader $ a+b $, der pr. definition er 1.
\end{proof}


\end{document}