\documentclass[../main.tex]{subfiles}


\begin{document}
\chapter{Methods}
\section{Grupper og tilstandsrum}
For at motivere sværhedsgraden af problemet, vil aspekter ved Rubiks terning blive beskrevet ud fra et gruppeteoretisk perspektiv. 

Rubiks terning har seks sider og 26 cubies i alt; seks midtercubies med én udadvendt overflade, 12 kantcubies med to flader og otte hjørnecubies med tre flader. Hver side kan roteres $\frac{\pi}{2}$ radianer med eller mod urets retning. Fastholdes orientationen af Rubiks terning, vil positionen af alle centercubies også holdes fast, uafhængigt af hvordan siderne roteres. Således vil man kunne navngive hver side ud fra dens orientation, og vi har nu front F, bag B, op O, ned N, højre H og venstre V. De otte omkringliggende cubies på hver side nummereres som vist på Figur (TODO: LAV EN FLOT FIGUR). 

Ud fra kvart-drejnings-semantikken kan man foretage 12 forskellige handlinger på Rubiks terning (13 hvis man medtager handlingen ikke at foretage nogen handling); hver side kan roteres én gang enten med eller mod uret. Notationsmæssigt betyder rotationen F, at fronten roteres $\frac{\pi}{2}$ radianer med urets retning. 2F svarer til en rotation af fronten på $\pi$ radianer, og 3F = F' svarer til én rotation mod urets retning. 

En rotation kan ses som en permutation af Rubiks terning. En permutation $f:A\rightarrow A$ en afbildning af mængden A på sig selv. En rotation kan skrives som en permutation bestående af fem disjunkte 4-cykler. Ud fra navngivningen i Figur (TODO: REF SAMME FIGUR SOM FØR) vil rotationen F kunne skrives som

$$F:(\begin{matrix}f_1 & f_7 & f_5 & f_3\end{matrix})
(\begin{matrix}f_2 & f_8 & f_6 & f_4\end{matrix})
(\begin{matrix}v_1 & o_3 & h_7 & n_5\end{matrix})
(\begin{matrix}o_2 & h_6 & n_4 & v_8\end{matrix})
(\begin{matrix}o_1 & h_5 & n_3 & v_7\end{matrix})$$

Ligeledes kan resten af permutationerne B, O, N, H og V hver skrives som en kombination af fem disjunkte 4-cykler. (TODO: ENTEN SKRIV DEM IND ELLER TILFØJ TIL BILAG) Ikke at foretage en rotation navngives I.

Mængden $\{F, B, O, N, H, V, I\}$ er en permutationsgruppe $(\mathbb{G}, *)$ med gruppeoperatoren $*$ defineret som følger; $X*Y$ er udførelsen af handling X efterfulgt af handling Y. $\mathbb{G}$ opfylder forudsætningerne for en permutationsgruppe: 
\begin{itemize}
%\item $\mathbb{G}$ er lukket, så hvis $X*Y=Z$ er Z også et element i $\mathbb{G}$. 
\item $\mathbb{G}$ er associativ, eftersom $(X*Y)*Z=X*(Y*Z)$; at foretage handling X og Y efterfulgt af Z er det samme som at foretage handling X efterfulgt af Y og Z. 
\item $\mathbb{G}$ har et identitetselement I (ikke at foretage en handling), eftersom $I*X = X = X*I$. 
\item Hvis X er et element i $\mathbb{G}$ har X en invers $X^{-1}$ således at $X*X^{-1}=I$. Dette er opfyldt, eftersom  $X^{-1}=X^3=X'$, altså at foretage samme handling mod urets retning.  
\end{itemize}

Med denne viden om $\mathbb{G}$ og Rubiks terning kan vi nu beregne antallet af permutationer (ordenen af $\mathbb{G}$). Først beregnes alle permutationer, både lovlige og ulovlige. Dette svarer til alle kombinationer, hvor man fysisk tager en cubie og placerer den, som man vil. Der er otte pladser til hjørnecubies, som hver har tre måder at blive orienteret på. Der er 12 pladser til kantcubies, som hver har to måder at blive orienteret på. Eftersom midtercubies er fikserede, fås 

$$8!\cdot3^8\cdot12!\cdot2^{12}=519.024.039.293.878.272.000=5,2\cdot10^{20}$$

permutationer. Der skal dog tages højde for ulovlige permutationer. Af disse er kun halvdelen af kantcubies placeret lovligt og en tredjedel af hjørnecubies placeret lovligt. Endvidere er der i halvdelen af permutationerne blevet udskiftet et ulige antal cubies, hvilket er umuligt. Derfor her vi

$$\frac{8!\cdot3^8\cdot12!\cdot2^{12}}{2\cdot3\cdot2}=43.252.003.274.489.856.000=4,3\cdot10^{19}$$

lovlige permutationer. Dette er ligeledes størrelsen på tilstandsrummet for problemet. 
 















 


\end{document}